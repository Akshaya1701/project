\documentclass{beamer}

\usetheme{AnnArbor}

%\title{WATER SANITATION AND CLEAN WATER.}
\title{Sri Sairam Engineering College, chennai}
\author{TEAM ID:1202}


%\textbf{\textcolor{cyan}{\\\underline{TEAM ID:}}}
 %\textit {\textcolor{magenta}{1202\\}}

%\textcolor{cyan}{\textbf{\\\underline{MENTOR:}}}
%\textit{\textcolor{magenta}{Arul Selvam D\\}}

%\textbf{\textcolor{cyan}{\\\underline{TEAM MEMBERS:}}}

%\textit{\textcolor{magenta}{Akshaya C\\ Annaprabu R\\  Amruta R\\ Kannammai M}}
%}

\begin{document}

\begin{frame}{    \   \   \    \   \    \   \   \    \   \    \    \    \   \   \      \    \   \      e-Yantra Ideas Competition\\     \textbf{\textcolor{blue}{     \   \   \    \   \    \   \   \    \   \    \    \    \   \   \      \  AUTOMATED TOILBOTS}}}






\textbf{\textcolor{blue}{PROJECT DOMAIN: Water Sanitation and Clean Water}}

\textbf{\textcolor{blue}{TEAM NO: \   \    \   \   \    \   \    \    \    \   \   \       1202}}

\textbf{\textcolor{blue}{COLLEGE: \   \   \      \   \   \   \    \    \     \         \       \     \    Sri Sairam Engineering College, Chennai.}}

\textcolor{blue}{\textbf{MENTOR:\   \    \   \   \    \   \    \    \    \   \   \     \   \      Arul Selvam D}}


\textbf{\textcolor{blue}{TEAM MEMBERS:   \    \         Akshaya C\\   \        \    \  \       \    \   \    \    \    \   \   \    \   \    \   \   \    \   \    \    \    \   \   \      \   \   \             Annaprabu R\\     \   \    \     \    \     \    \    \   \   \    \    \   \   \  \  \   \    \    \     \   \    \    \   \   \   \     \         Amruta R\\     \   \         \    \   \    \    \    \   \   \    \   \       \   \    \    \   \    \   \    \    \       \      \    \     \   \   \    Kannammai M}}





\end{frame}


\begin{frame}
{\underline{\textbf{INDEX:}}}

\begin{itemize}
\item \underline{\textbf{\textcolor{magenta}{Table Of Contents:}}}
\begin{enumerate}
\begin{columns}

\column{0.2\textwidth}
\item \textit{\textcolor{blue}{Introduction}}



\item \textit{\textcolor{blue}{Implementation}}


\item \textit{\textcolor{blue}{Flow Chart}}

\item \textit{\textcolor{blue}{Illustration}}
\column{0.45\textwidth}

\item \textit{\textcolor{blue}{Electrical Components}}


\item \textit{\textcolor{blue}{Advantages}}

\item \textit{\textcolor{blue}{Improvements}}

\item \textit{\textcolor{blue}{Conclusion}}

\end{columns}
\end{enumerate}
\end{itemize}
\end{frame}


\begin{frame}{\underline{\textbf{INTRODUCTION:}}}
\textit{\textcolor{magenta}{"Sanitation is more important than independence"}}\textcolor{blue}{ as per Gandhiji. India's battle with total sanitation is an ongoing saga with successive governments working on it. Following the Gandhian ideal of }\textit{\textcolor{magenta}{ "Sanitation for all", }}\textcolor{blue}{the Government of India launched Clean India campaign popularly known as} \textit{\textcolor{magenta}{"Swach Bharat Mission" }}\textcolor{blue}{on Oct 2,2014. As a part of this mission, more than 92\% households were provided with toilets by the year 2018.The Swach bharat abhiyan is not just done with building toilets.  In next three years, NDA will take up maintenance works at nearly 1,28,000 toilets built at public places. In compliance with Swach bharat abhiyan mission,  Automatic Toilbot is proposed. Moreover, the conventional maintenance works are being carried out manually which causes spread of infections to the individuals involved. To overcome this issue, the bot will clean the toilets  automatically by its robotic arms.}
\end{frame}


\begin{frame}
{\underline{\textbf{IMPLEMENTATION:}}}

\textcolor{blue}{The automated toilbot is an automatic toilet cleaning machine.
It is first turned on by the user. Indicator lights indicate the status of the bot. Then the user is asked for the input details like the type of toilet , number of toilets , distance between the toilets and distance to the toilets (approximated value) and the bot reaches the toilet according to the provided inputs.Then the robotic arm’s height is  adjusted depending on the type of toilet. Next, the cleaning operation starts by pouring water. The water is supplied through pipes and its flow is controlled using solenoid valves. Then the cleaning agent which is initially stored is also poured by robotic arm movements and the brush action is performed. Eventually,the toilet is rinsed with water and this process continues till the bot reaches the last toilet.}


\end{frame}


\begin{frame}
{\underline{\textbf{FLOW CHART:}}}

\begin{figure}\begin{columns}

\column{0.5\textwidth}
\includegraphics[width=1.2\linewidth]{flowchart.jpeg}


\column{0.5\textwidth}
\includegraphics[width=1.2\linewidth]{flow1.jpeg}
\end{columns}
\end{figure}


\end{frame}

\begin{frame}
{\underline{\textbf{ILLUSTRATION:}}}
\textbf{\textcolor{blue}{\underline{Arm Movements:}}}
\begin{figure}\begin{columns}

\column{0.5\textwidth}
\includegraphics[width=0.7\linewidth]{indiantoil.jpeg}
\caption{Indian model  \   \   \    \    \   \   \    \    }


\column{0.5\textwidth}
\includegraphics[width=0.7\linewidth]{westerntoil.jpeg}
\caption{Western model  \   \   \    \    \   \   \    \    }
\end{columns}
\end{figure}

\end{frame}

\begin{frame}
{\underline{\textbf{ILLUSTRATION:}}}

\textbf{\textcolor{blue}{\underline{Water Flow \& Brush Cleaning:}}}
\begin{figure}\begin{columns}

\column{0.5\textwidth}

\includegraphics[width=0.7\linewidth]{waterflow.jpeg}
\caption{ Water Flow  \   \   \    \    \   \   \    \    }

\column{0.5\textwidth}
\includegraphics[width=0.7\linewidth]{cleaning.jpeg}
\caption{Brush Cleaning  \   \   \   \    \    \   \   \    \    }
\end{columns}
\end{figure}

\end{frame}


\begin{frame}{\underline{\textbf{ELECTRICAL COMPONENTS:}}}

\textbf{\textcolor{blue}{1. Robotic arm\\
2. 4 cross 4 keyboard matrix\\
3. 20 cross 4 Lcd \\
4. Indicator light\\
5. Ultrasonic sensor\\
6. Rubber wheels\\
7. Motor\\
8. Motor drive\\
9. Solenoid valves
}}
\end{frame}

\begin{frame}{\underline{\textbf{ADVANTAGES AND IMPROVEMENTS:}}}



\textbf{\textcolor{magenta}{Advantages:\\}}
\textcolor{blue}{1 Dynamical - The bot can move over ah row of toilets and clean\\  \   \  \   them.\\
2 Flexibility -  The bot can be used for both Western and Indian\\    \   \  \  
 style toilets.\\
3 Easy design and simple usage.\\}



\textbf{\textcolor{magenta}{Improvements:\\}}
\textcolor{blue}{
1. Automatic detection of toilets.\\
2. cleaning up the entire the washroom floor.\\
3. Identification of germs.}

\end{frame}

\begin{frame}{\underline{\textbf{CONCLUSION:}}}


\textbf{\textcolor{blue}{The recent WHO report states that just because of cleanliness, three lakh children could be saved from dying in India, such is the situation. Thus, the automated toilbot is been proposed to prevent loss of lives and provide a hygienic, disease free nation for the future generation.
}}
\end{frame}











\begin{frame}{\underline{\textbf{References:}}}

[1] Newspaper articles\\a)\textcolor{blue}{https://www.google.com/amp/s/www.downtoearth.org.in/blog/waste\\/amp/world-toilet-day-why-future-of-swachh-bharat-mission-remains-unsure-62166\\}

b) \textcolor{blue}{https://www.thebetterindia.com/13940/etoilet-changing-way-public-sanitation-works-india-eram-marico/\\}

c)\textcolor{blue}{https://www.google.com/amp/s/www.livemint.com/Politics/\\3WS67hUlMgT4SDrBTShESI/Govt-working-on-maintenance-plans-for-toilets-made-under-Swa.html\%3ffacet=amp\\}

[2] Mrs K Elavarasi, Mrs Suganthi and Mrs Jayachitra, "Developing Smart Toilets using IOT" of IFET college of engineering, Villupuram, Tamil Nadu.\\

[3] Anil C Gawande, Shivani, Ashwin Satone and Prasad Kade's " Design and fabrication of Advanced mechanism for Indian toilet dome cleaning with multiwasher assembly", Assistant  Professor and student of D.M.I.E.T.R, Wardha.\\




\end{frame}
















\end{document}